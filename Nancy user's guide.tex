% Nancy User’s Guide

\documentclass[english]{scrartcl}
\usepackage{babel,a4,newlfont,fancyvrb,copyrght}
\usepackage[utf8]{inputenc}

% Alter some default parameters for general typesetting
\frenchspacing

% New commands
\renewcommand{\copyrightyear}{2002--2010}

\begin{document}

\title{Nancy, the lazy web site maker\\User’s guide}
\date{14th August 2010}
\author{Reuben Thomas}
\maketitle
% FIXME: Add logo

\section{Introduction}

Nancy is a simple web site maker that glues together HTML fragments and other files to make pages, and finds all the pages and files that make up a site. Files can be generated programmatically, and specialized for particular pages.

Nancy has a command-line tool to build a static site and a CGI script to generate pages on the fly.
% FIXME: Document webnancy

\section{Invocation}

Nancy takes four arguments:

\begin{verbatim}
nancy.pl [OPTIONS] SOURCES DESTINATION TEMPLATE [HOME]
\end{verbatim}

\noindent where \verb|SOURCES| is a list of directories that contain the source (separated by the system directory separator, usually a colon), \verb|DESTINATION| is the directory to which the resulting files will be written, \verb|TEMPLATE| is the name of the template file, and \verb|HOME| gives the home page of the site, which defaults to \verb|index.html|.

The following command-line options may be given:

\begin{description}
\SaveVerb{listfiles}|--list-files|
\item[\UseVerb{listfiles}]List on standard error the files used to make each page.
\SaveVerb{version}|--version|
\item[\UseVerb{version}]Print the version number of Nancy.
\SaveVerb{help}|--help|
\item[\UseVerb{help}]Print usage information for Nancy.
\end{description}

The options may be abbreviated to any unambiguous prefix.

\section{Operation}
\label{operation}

Nancy produces the site according to the following algorithm:

\begin{enumerate}
\item Start with the \verb|HOME|.
\item For the current object:
\begin{enumerate}
\item If it is a file, copy it to the output.
\item Otherwise, assuming it is a directory:
\begin{enumerate}
\item Set the initial text to \verb|$include{TEMPLATE}|.
\item Repeatedly scan the text for a command and replace it by its output, until no more commands are found.
\item Find all the \verb|href| and \verb|src| links in the text, and process each as the current object.
\item Write the resultant text to a file: for each directory \verb|SOURCE/PATH| the output file is \verb|DESTINATION/PATH|.
\end{enumerate}
\end{enumerate}
\end{enumerate}

In general, only leaf directories, that is, directories that only contain files, should correspond to pages: this is to ensure that every page can be specialised without affecting any other page. It is advisable to ensure that every non-leaf directory has an \verb|index.html| sub-directory (or some other valid index page name), so that there are no URLs in the resulting site that do not correspond to a page.

A command takes the form

\begin{verbatim}
$COMMAND{ARGUMENT, ...}
\end{verbatim}

Nancy recognises these commands:

\begin{description}
\SaveVerb{include}|$include{FILE}|
\item[\UseVerb{include}]Replace the command with the contents of the given file.
\SaveVerb{root}|$root{}|
\SaveVerb{rellink}|<a href="root{}/path/to/page.html">|
\item[\UseVerb{root}]Replace the command with the relative URL to the root of the site from the page under construction. This means that every link in a site can be written relative to the current page, either explicitly (which is a good way to link to pages related to the current page, as such links do not need to be rewritten if the related pages are moved together within the site), or implicitly as \UseVerb{rellink}. Hence the site’s base URL can be changed without needing to change any intra-site links.
\SaveVerb{run}|$run{FILE[, ARGUMENT, ...]}|
\item[\UseVerb{run}]Replace the command with the output of the given
file evaluated as a Perl expression, which is expected to produce a subroutine, which is then called with the given arguments, and the directory in which the current file is found as an extra final argument, represented as a reference to an array of path components, left-to-right.
\end{description}

Only one guarantee is made about the order in which commands are processed: if one command is nested inside another, the inner command will be processed first. (The order only matters for \verb|$run| commands; if you nest them, you have to deal with this potential pitfall.)

To find the file \verb|FILE| specified by a \verb|$include| or \verb|$run| command, Nancy proceeds thus:

\begin{itemize}
\item Consider the source trees in right-to-left order. For each tree:
\begin{enumerate}
\item See whether \verb|SOURCE/PAGE_PATH/FILE| exists. If the file exists but is empty, or any directory in \verb|FILE|’s path is empty, consider \verb|FILE| not to have been found.
\item If the file is not found, remove the last directory from \verb|PAGE_PATH| and try again, until \verb|PAGE_PATH| is empty.
\item Finally, try looking in \verb|SOURCE/FILE|.
\end{enumerate}
\end{itemize}

For example, if \verb|SOURCES| is \verb|/dir| and \verb|FILE| is \verb|foo/bar/baz|, and Nancy is trying to find \verb|file.html|, it will try the following directories, in order:

\begin{enumerate}
\item \verb|/dir/foo/bar/baz/file.html|
\item \verb|/dir/foo/bar/file.html|
\item \verb|/dir/foo/file.html|
\item \verb|/dir/file.html|
\end{enumerate}

\section{Example}
% FIXME: add example use of $root{}: a link from a Person to a Place
% FIXME: add example use of $run{}
% FIXME: illustrate the use of multiple source trees (based on the
% tests)
% FIXME: the examples are unclear; really ought to be actual web pages
% with some sort of structure diagrams automatically generated

Suppose a web site has the following page design, from top to bottom: logo, navigation menu, breadcrumb trail, page body.

Most of the elements are the same on each page, but the breadcrumb trail has to show the canonical path to each page, and the logo is bigger on the home page, which is the default \verb|index.html|.

Suppose further that the web site has the following structure, where each line corresponds to a page:

\begin{itemize}
\item Home page
\item People
  \begin{itemize}
  \item Jo Bloggs
  \item Hilary Pilary
  \item \dots
  \end{itemize}
\item Places
  \begin{itemize}
  \item Vladivostok
  \item Timbuktu
  \item \dots
  \end{itemize}
\end{itemize}

The basic page template looks something like this:

\begin{verbatim}
<html>
  <link href="style.css" rel="stylesheet" type="text/css">
  <title>$include{title}</title>
  <body>
    <div class="logo">$include{logo.html}</div>
    <div class="menu">$include{menu.html}</div>
    <div class="breadcrumb">$include{breadcrumb.html}</div>
    <div class="main">$include{main.html}</div>
  </body>
</html>
\end{verbatim}

Making the menu an included file is not strictly necessary, but makes the template easier to read. The pages will be laid out as follows:

\begin{itemize}
\item \verb|/|
  \begin{itemize}
  \item \verb|index.html|
  \item \verb|people/|
    \begin{itemize}
    \item \verb|index.html|
    \item \verb|jo_bloggs.html|
    \item \verb|hilary_pilary.html|
    \end{itemize}
  \item \verb|places/|
    \begin{itemize}
    \item \verb|index.html|
    \item \verb|vladivostok.html|
    \item \verb|timbuktu.html|
    \end{itemize}
  \end{itemize}
\end{itemize}

The corresponding source files will be laid out as follows. This may look a little confusing at first, but note the similarity to the HTML pages, and hold on for the explanation!

\begin{itemize}
\item \verb|source/|
  \begin{itemize}
  \item \verb|template.html| (the template shown above)
  \item \verb|menu.html|
  \item \verb|logo.html|
  \item \verb|breadcrumb.html|
  \item \verb|index.html/|
    \begin{itemize}
    \item \verb|main.html|
    \item \verb|logo.html|
    \end{itemize}
  \item \verb|people/|
    \begin{itemize}
    \item \verb|breadcrumb.html|
    \item \verb|index.html/|
      \begin{itemize}
      \item \verb|main.html|
      \end{itemize}
    \item \verb|jo_bloggs.html/|
      \begin{itemize}
      \item \verb|main.html|
      \end{itemize}
    \item \verb|hilary_pilary.html/|
      \begin{itemize}
      \item \verb|main.html|
      \end{itemize}
    \end{itemize}
  \item \verb|places/|
    \begin{itemize}
    \item \verb|breadcrumb.html|
    \item \verb|index.html/|
      \begin{itemize}
      \item \verb|main.html|
      \end{itemize}
    \item \verb|vladivostok.html/|
      \begin{itemize}
      \item \verb|main.html|
      \end{itemize}
    \item \verb|timbuktu.html/|
      \begin{itemize}
      \item \verb|main.html|
      \end{itemize}
    \end{itemize}
  \end{itemize}
\end{itemize}

Note that there is only one menu fragment (the main menu is the same for every page), while each section has its own breadcrumb trail (\verb|breadcrumb.html|), and each page has its own content (\verb|main.html|).

To build the site, Nancy is invoked as:

\begin{verbatim}
nancy.pl source template.html dest
\end{verbatim}

Now consider how Nancy builds the page whose URL is \verb|vladivostok.html|. According to the rules given in Section~\ref{operation}, Nancy will look first for files in \verb|source/places/vladivostok.html|, then in \verb|source/places|, and finally in \verb|source|. Hence, the actual list of files used to assemble the page is:

\begin{itemize}
\item \verb|source/template.html|
\item \verb|source/logo.html|
\item \verb|source/menu.html|
\item \verb|source/places/breadcrumb.html|
\item \verb|source/places/vladivostok.html/main.html|
\end{itemize}

For the site’s index page, the file \verb|index.html/logo.html| will be used for the logo fragment, which can refer to the larger graphic desired.

This scheme, though simple, is surprisingly flexible; this simple example has covered all the standard techniques for Nancy’s use.

\end{document}

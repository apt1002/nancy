% Nancy User’s Guide

\documentclass[english]{scrartcl}
\usepackage{babel,a4,newlfont,fancyvrb,copyrght}
\usepackage[utf8]{inputenc}

% Alter some default parameters for general typesetting
\frenchspacing

% New commands
\renewcommand{\copyrightyear}{2002--2011}

\begin{document}

\title{Nancy, the lazy web site maker\\User’s guide}
\date{4th November 2011}
\author{Reuben Thomas}
\maketitle
% FIXME: Add logo

\section{Introduction}

Nancy is a simple web site maker that finds and glues together HTML fragments and other files to make pages, and serves them and all the static files (CSS, images, etc.) that make up the site. Files can be generated programmatically, and specialized for particular pages.

Nancy is run as a CGI program by a web server.

\section{Invocation}

Nancy is configured via environment variables:

\begin{description}
\SaveVerb{webroot}|NANCY_WEB_ROOT|
\item[\UseVerb{webroot}]the URL of the root of the site
\SaveVerb{sources}|NANCY_SOURCES|
\item[\UseVerb{sources}]a list of directories that contain the source (separated by the system directory separator, usually a colon)
\SaveVerb{template}|NANCY_TEMPLATE|
\item[\UseVerb{template}]the template name, which defaults to \verb|template|
\SaveVerb{index}|NANCY_INDEX|
\item[\UseVerb{index}]the index page, which defaults to \verb|index.html|
\SaveVerb{listfiles}|NANCY_LIST_FILES|
\item[\UseVerb{listfiles}]if defined, make Nancy list on standard error the files used to make each page
\end{description}

\section{Operation}
\label{operation}

Nancy serves a URL as follows:

\begin{enumerate}
\item Remove the \verb|NANCY_WEB_ROOT| prefix from the URL, convert the rest of the URL to a file path and look up the path in the sources.
\item If it is a file, output it.
\item Otherwise, assuming it is a directory \verb|foo|:
\begin{enumerate}
\item If the directory’s name ends in an extension \verb|.bar|, that is, a period followed by one or more word characters:
\begin{enumerate}
\item Set the initial text to \verb|$include{TEMPLATE.bar}|.
\item Repeatedly scan the text for a command and replace it by its output, until no more commands are found.
\item Output the resultant text.
\end{enumerate}
\item Otherwise, try to serve the URL obtained by concatenating the directory separator followed by \verb|NANCY_INDEX| to the URL.
\end{enumerate}

To look up a path in the sources, Nancy does the following:

\begin{enumerate}
\item For each directory in \verb|NANCY_SOURCES|, in right-to-left order, prepend it to the path.
\item Look up the path; if it yields a file or directory, stop, with that as the result (symbolic links are dereferenced).
\item If no file or directory is found, fail.
\end{enumerate}

In general, only leaf directories, that is, directories that only contain files, should correspond to pages: this is to ensure that each page can be specialised without affecting any other page. It is advisable to ensure that every non-leaf directory has a sub-directory whose name is the value of \verb|NANCY_INDEX|, so that there are no valid URLs in the resulting site that do not correspond to a page.

A command takes the form

\begin{verbatim}
$COMMAND{ARGUMENT, ...}
\end{verbatim}

Nancy recognises these commands:

\begin{description}
\SaveVerb{include}|$include{FILE}|
\item[\UseVerb{include}]Replace the command with the contents of the given file.
\SaveVerb{root}|$root{}|
\SaveVerb{rellink}|<a href="$root{}/path/to/page.html">|
\item[\UseVerb{root}]Replace the command with the relative URL to the root of the site from the page under construction. This means that every link in a site can be written relative to the current page, either explicitly (which is a good way to link to pages related to the current page, as such links do not need to be rewritten if the related pages are moved together within the site), or implicitly as \UseVerb{rellink}. Hence the site’s root URL can be changed without needing to change any intra-site links, and the site need not be at the root of its host (as would be necessary with a link starting with a forward slash).
\SaveVerb{run}|$run{FILE[, ARGUMENT, ...]}|
\item[\UseVerb{run}]Replace the command with the output of the given
file evaluated as a Perl expression, which is expected to produce a subroutine, which is then called with the given arguments, followed by the path of the page currently being expanded, represented as a reference to an array of path components, left-to-right, and a list of the source tree roots, each represented in the same way, one extra argument per root.
\end{description}

Only one guarantee is made about the order in which commands are processed: if one command is nested inside another, the inner command will be processed first. (The order only matters for \verb|$run| commands; if you nest them, you have to deal with this potential pitfall.)

To find the file \verb|FILE| specified by a \verb|$include| or \verb|$run| command, Nancy proceeds thus:

\begin{enumerate}
\item Look up \verb|PAGE_PATH/FILE| in the sources.
\item If the file is not found, remove the last directory from \verb|PAGE_PATH| and try again, until \verb|PAGE_PATH| is empty.
\item Finally, try looking for \verb|FILE|.
\end{enumerate}

For example, if \verb|SOURCES| is \verb|/dir| and \verb|FILE| is \verb|foo/bar/baz|, and Nancy is trying to find \verb|file.html|, it will try the following directories, in order:

\begin{enumerate}
\item \verb|/dir/foo/bar/baz/file.html|
\item \verb|/dir/foo/bar/file.html|
\item \verb|/dir/foo/file.html|
\item \verb|/dir/file.html|
\end{enumerate}

\section{Example}
% FIXME: add example use of $root{}: a link from a Person to a Place
% FIXME: add example use of $run{}
% FIXME: illustrate the use of multiple source trees (based on the tests)
% FIXME: the examples are unclear; really ought to be actual web pages with some sort of structure diagrams automatically generated

Suppose a web site has the following page design, from top to bottom: logo, navigation menu, breadcrumb trail, page body.

Most of the elements are the same on each page, but the breadcrumb trail has to show the canonical path to each page, and the logo is bigger on the home page, which is the default \verb|index.html|.

Suppose further that the web site has the following structure, where each line corresponds to a page:

\begin{itemize}
\item Home page
\item People
  \begin{itemize}
  \item Jo Bloggs
  \item Hilary Pilary
  \item \dots
  \end{itemize}
\item Places
  \begin{itemize}
  \item Vladivostok
  \item Timbuktu
  \item \dots
  \end{itemize}
\end{itemize}

The basic page template looks something like this:

\begin{verbatim}
<html>
  <link href="style.css" rel="stylesheet" type="text/css">
  <title>$include{title}</title>
  <body>
    <div class="logo">$include{logo.html}</div>
    <div class="menu">$include{menu.html}</div>
    <div class="breadcrumb">$include{breadcrumb.html}</div>
    <div class="main">$include{main.html}</div>
  </body>
</html>
\end{verbatim}

Making the menu an included file is not strictly necessary, but makes the template easier to read. The pages will be laid out as follows:

\begin{itemize}
\item \verb|/|
  \begin{itemize}
  \item \verb|index.html|
  \item \verb|people/|
    \begin{itemize}
    \item \verb|index.html|
    \item \verb|jo_bloggs.html|
    \item \verb|hilary_pilary.html|
    \end{itemize}
  \item \verb|places/|
    \begin{itemize}
    \item \verb|index.html|
    \item \verb|vladivostok.html|
    \item \verb|timbuktu.html|
    \end{itemize}
  \end{itemize}
\end{itemize}

The corresponding source files will be laid out as follows. This may look a little confusing at first, but note the similarity to the HTML pages, and hold on for the explanation!

\begin{itemize}
\item \verb|source/|
  \begin{itemize}
  \item \verb|template.html| (the template shown above)
  \item \verb|menu.html|
  \item \verb|logo.html|
  \item \verb|breadcrumb.html|
  \item \verb|index.html/|
    \begin{itemize}
    \item \verb|main.html|
    \item \verb|logo.html|
    \end{itemize}
  \item \verb|people/|
    \begin{itemize}
    \item \verb|breadcrumb.html|
    \item \verb|index.html/|
      \begin{itemize}
      \item \verb|main.html|
      \end{itemize}
    \item \verb|jo_bloggs.html/|
      \begin{itemize}
      \item \verb|main.html|
      \end{itemize}
    \item \verb|hilary_pilary.html/|
      \begin{itemize}
      \item \verb|main.html|
      \end{itemize}
    \end{itemize}
  \item \verb|places/|
    \begin{itemize}
    \item \verb|breadcrumb.html|
    \item \verb|index.html/|
      \begin{itemize}
      \item \verb|main.html|
      \end{itemize}
    \item \verb|vladivostok.html/|
      \begin{itemize}
      \item \verb|main.html|
      \end{itemize}
    \item \verb|timbuktu.html/|
      \begin{itemize}
      \item \verb|main.html|
      \end{itemize}
    \end{itemize}
  \end{itemize}
\end{itemize}

Note that there is only one menu fragment (the main menu is the same for every page), while each section has its own breadcrumb trail (\verb|breadcrumb.html|), and each page has its own content (\verb|main.html|).

Now consider how Nancy builds the page whose URL is \verb|vladivostok.html|. According to the rules given in section~\ref{operation}, Nancy will look first for files in \verb|source/places/vladivostok.html|, then in \verb|source/places|, and finally in \verb|source|. Hence, the actual list of files used to assemble the page is:

\begin{itemize}
\item \verb|source/template.html|
\item \verb|source/logo.html|
\item \verb|source/menu.html|
\item \verb|source/places/breadcrumb.html|
\item \verb|source/places/vladivostok.html/main.html|
\end{itemize}

For the site’s index page, the file \verb|index.html/logo.html| will be used for the logo fragment, which can refer to the larger graphic desired.

This scheme, though simple, is surprisingly flexible; this simple example has covered all the standard techniques for Nancy’s use.

\end{document}

% Nancy User's Guide

\documentclass[english]{scrartcl}
\usepackage{babel,a4,newlfont,fancyvrb,copyrght}
\usepackage[utf8]{inputenc}

% Alter some default parameters for general typesetting
\frenchspacing

% New commands
\renewcommand{\copyrightyear}{2002--2010}

\begin{document}

\title{Nancy, the lazy web site builder\\User's guide}
\date{8th April 2010}
\author{Reuben Thomas}
\maketitle

\section{Introduction}

Nancy is a simple web site builder that glues together HTML and other fragments to make pages, and allows fragments to be specialised for particular pages. Fragments can be files or generated programmatically.

Nancy is a command-line tool.

\section{Invocation}

Nancy takes four arguments:

\begin{verbatim}
nancy.pl [OPTIONS] SOURCES DESTINATION TEMPLATE START
\end{verbatim}

\noindent where \verb|SOURCES| is a colon-separated list of directories that contain the source, \verb|DESTINATION| is the directory to which the resulting HTML pages will be written, \verb|TEMPLATE| is the name of the template file, and \verb|START| gives the top-most page of the site.

The following command-line options may be given to Nancy:

\begin{description}
\SaveVerb{listfiles}|--list-files|
\item[\UseVerb{listfiles}]List on standard error the fragments used to make each page, and insert a message into the output directly before the contents of the include giving the file name of the fragment.
\SaveVerb{warn}|--warn|
\item[\UseVerb{warn}]Print various warnings about possible problems with the fragments, including listing unused and duplicate fragments.
\SaveVerb{version}|--version|
\item[\UseVerb{version}]Print the version number of Nancy.
\SaveVerb{help}|--help|
\item[\UseVerb{help}]Print usage information for Nancy.
\end{description}

The options may be abbreviated to any unambiguous prefix.

\section{Operation}
\label{operation}

Nancy produces the finished pages according to the following algorithm:

% FIXME: Need to formalise how image &c. files get into the output
\begin{enumerate}
\item Scan the source trees from right to left, adding the directories that appear in each tree to a list. If a given directory is empty, remove it and its contents from the list.
\item Add the page \verb|START| to an empty list.
\item While there is at least one previously unprocessed page in the list that corresponds to a directory in the input, take one and:
\begin{enumerate}
\item Set the initial text to \verb|$include{TEMPLATE}|.
\item Repeatedly scan the text for a command and replace it by its output, until no more commands are found.
\item Find all the \verb|href| links in the text, and add them to the list.
\item Write the resultant text to a file: for each directory \verb|SOURCE/PATH| the output file is \verb|DESTINATION/PATH|.
\end{enumerate}
\end{enumerate}

In general, only leaf directories, that is, directories that only contain files, should correspond to pages: this is to ensure that every page can be specialised without affecting any other page. It is advisable to ensure that every non-leaf directory has an \verb|index.html| sub-directory (or some other valid index page name), so that there are no URLs in the resulting site that do not correspond to a page.

A command takes the form

\begin{verbatim}
$COMMAND{ARGUMENT, ...}
\end{verbatim}

Nancy recognises these commands:

\begin{description}
\SaveVerb{include}|$include{FILE}|
\item[\UseVerb{include}]Replace the command with the contents of the given file.
\SaveVerb{page}|$page{}|
\item[\UseVerb{page}]Replace the command with the URL of the page
under construction, relative to the root of the site.
\SaveVerb{root}|$root{}|
\SaveVerb{rellink}|<a href="root{}/path/to/page.html">|
\item[\UseVerb{root}]Replace the command with the relative URL to the root of the site from the page under construction. This means that every link in a site can be written relative to the current page, either explicitly (which is a good way to link to pages related to the current page, as such links do not need to be rewritten if the related pages are moved together within the site), or implicitly as \UseVerb{rellink}. Hence the site's base URL can be changed without needing to change any intra-site links.
\SaveVerb{run}|$run{PERL-FRAGMENT[, ARGUMENT, ...]}|
\item[\UseVerb{run}]Replace the command with the output of the given
fragment evaluated as a Perl expression, which is expected to produce a subroutine, which is then called with the given arguments.
\end{description}

Only one guarantee is made about the order in which commands are processed: if one command is nested inside another, the inner command will be processed first. (The order only matters for |\verb|$run| commands; if you nest them, you have to deal with this potential pitfall.)

To find the fragment \verb|FRAGMENT_PATH| specified by an \verb|$include| command, Nancy proceeds thus:

\begin{itemize}
\item Consider the source trees in left-to-right order:
\begin{enumerate}
\item Look in \verb|SOURCE/PAGE_PATH/FRAGMENT_PATH|.
\item If the file is not found, remove the last directory from \verb|PAGE_PATH| and try again, until \verb|PAGE_PATH| is empty.
\item Finally, try looking in \verb|SOURCE/FRAGMENT_PATH|.
\end{enumerate}
\end{itemize}

For example, if \verb|SOURCES| is \verb|/dir| and \verb|FILE_PATH| is \verb|foo/bar/baz|, and Nancy is trying to find \verb|file.html|, it will try the following directories, in order:

\begin{enumerate}
\item \verb|/dir/foo/bar/baz/file.html|
\item \verb|/dir/foo/bar/file.html|
\item \verb|/dir/foo/file.html|
\item \verb|/dir/file.html|
\end{enumerate}

\section{Example}
% FIXME: add use of $root{}: a link from a Person to a Place
% FIXME: add use of $page{} and $run{} (together? I don't currently
% have another use for $page{})
% FIXME: illustrate the use of multiple source trees (based on the
% tests)
% FIXME: the examples are unclear; really ought to be actual web pages % with some sort of structure diagrams automatically generated

Suppose a web site has the following page design, from top to bottom: logo, navigation menu, breadcrumb trail, page body.

Most of the elements are the same on each page, but the breadcrumb trail has to show the canonical path to each page, and the logo is bigger on the home page, which is called \verb|index.html|.

Suppose further that the web site has the following structure, where each line corresponds to a page:

\begin{itemize}
\item Home page
\item People
  \begin{itemize}
  \item Jo Bloggs
  \item Hilary Pilary
  \item \dots
  \end{itemize}
\item Places
  \begin{itemize}
  \item Vladivostok
  \item Timbuktu
  \item \dots
  \end{itemize}
\end{itemize}

The basic page template looks something like this:

\begin{verbatim}
<html>
  <link href="style.css" rel="stylesheet" type="text/css">
  <title>$include{title}</title>
  <body>
    <div class="logo">$include{logo.html}</div>
    <div class="menu">$include{menu.html}</div>
    <div class="breadcrumb">$include{breadcrumb.html}</div>
    <div class="main">$include{main.html}</div>
  </body>
</html>
\end{verbatim}

Making the menu an included file is not strictly necessary, but makes the HTML fragments easier to read. The pages will be laid out as follows:

\begin{itemize}
\item \verb|/|
  \begin{itemize}
  \item \verb|index.html|
  \item \verb|people/|
    \begin{itemize}
    \item \verb|index.html|
    \item \verb|jo_bloggs.html|
    \item \verb|hilary_pilary.html|
    \end{itemize}
  \item \verb|places/|
    \begin{itemize}
    \item \verb|index.html|
    \item \verb|vladivostok.html|
    \item \verb|timbuktu.html|
    \end{itemize}
  \end{itemize}
\end{itemize}

The corresponding source files will be laid out as follows. This may look a little confusing at first, but note the similarity to the HTML pages, and hold on for the explanation!

\begin{itemize}
\item \verb|source/|
  \begin{itemize}
  \item \verb|template.html| (the template shown above)
  \item \verb|menu.html|
  \item \verb|logo.html|
  \item \verb|breadcrumb.html|
  \item \verb|index.html/|
    \begin{itemize}
    \item \verb|main.html|
    \item \verb|logo.html|
    \end{itemize}
  \item \verb|people/|
    \begin{itemize}
    \item \verb|breadcrumb.html|
    \item \verb|index.html/|
      \begin{itemize}
      \item \verb|main.html|
      \end{itemize}
    \item \verb|jo_bloggs.html/|
      \begin{itemize}
      \item \verb|main.html|
      \end{itemize}
    \item \verb|hilary_pilary.html/|
      \begin{itemize}
      \item \verb|main.html|
      \end{itemize}
    \end{itemize}
  \item \verb|places/|
    \begin{itemize}
    \item \verb|breadcrumb.html|
    \item \verb|index.html/|
      \begin{itemize}
      \item \verb|main.html|
      \end{itemize}
    \item \verb|vladivostok.html/|
      \begin{itemize}
      \item \verb|main.html|
      \end{itemize}
    \item \verb|timbuktu.html/|
      \begin{itemize}
      \item \verb|main.html|
      \end{itemize}
    \end{itemize}
  \end{itemize}
\end{itemize}

Note that there is only one menu fragment (the main menu is the same for every page), while each section has its own breadcrumb trail (\verb|breadcrumb.html|), and each page has its own content (\verb|main.html|).

To build the site, Nancy is invoked as:

\begin{verbatim}
nancy.pl source template.html dest index.html
\end{verbatim}

Now consider how Nancy builds the page whose URL is \verb|vladivostok.html|. According to the rules given in Section~\ref{operation}, Nancy will look first for files in \verb|source/places/vladivostok.html|, then in \verb|source/places|, and finally in \verb|source|. Hence, the actual list of files used to assemble the page is:

\begin{itemize}
\item \verb|source/template.html|
\item \verb|source/logo.html|
\item \verb|source/menu.html|
\item \verb|source/places/breadcrumb.html|
\item \verb|source/places/vladivostok.html/main.html|
\end{itemize}

For the site's index page, the file \verb|index.html/logo.html| will be used for the logo fragment, which can refer to the larger graphic desired.

This scheme, though simple, is surprisingly flexible; this simple example has covered all the standard techniques for Nancy's use.

\end{document}
